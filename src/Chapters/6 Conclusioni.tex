\chapter{Conclusioni}
In questa tesi è stata condotta un’analisi sperimentale del fuzzing applicato a software reale, utilizzando il fuzzer AFL all’interno del framework MAGMA. L’obiettivo principale del lavoro era valutare l’efficacia del fuzzing nel generare input malformati ma sintatticamente validi, capaci di attivare vulnerabilità reali all’interno di un programma ampiamente utilizzato come \textit{libpng}.
I risultati ottenuti mostrano come AFL sia in grado di esplorare in modo progressivo ed efficace lo spazio degli input, partendo da un corpus iniziale estremamente ridotto e ampliandolo fino a generare un insieme di oltre un migliaio di file PNG differenti. L’evoluzione del corpus e l’elevato numero di test eseguiti evidenziano la capacità del fuzzer di adattare le proprie mutazioni sulla base del feedback di copertura, raggiungendo porzioni di codice non inizialmente previste dagli sviluppatori.
La campagna di fuzzing ha portato all’individuazione di diversi crash, tra i quali è stato selezionato un caso rappresentativo per un’analisi approfondita. La Root Cause Analysis ha permesso di ricondurre il crash a una vulnerabilità reale documentata, associata alla \texttt{CVE-2013-6954}, confermando la validità del benchmark MAGMA come strumento di valutazione basato su bug reali. L’analisi ha evidenziato come input semanticamente inconsistenti, pur rispettando la struttura sintattica del formato PNG, possano violare assunzioni implicite del codice e causare accessi non validi alla memoria.
Questo risultato sottolinea uno degli aspetti più rilevanti del fuzzing: la capacità di mettere in luce debolezze logiche e carenze di validazione che difficilmente emergerebbero tramite test manuali o casi di test tradizionali. In particolare, il caso analizzato mostra come la mancanza di controlli semantici su strutture dati critiche possa condurre a vulnerabilità anche in librerie mature e ampiamente utilizzate.
Il lavoro svolto presenta tuttavia alcuni limiti. L’analisi è stata condotta su un singolo fuzzer e un unico programma target, l’approfondimento tramite RCA è stato applicato a un solo crash rappresentativo. Estensioni future potrebbero includere il confronto tra diversi fuzzer, l’analisi automatizzata di un numero maggiore di crash e l’applicazione della metodologia a ulteriori formati di file o librerie.
In conclusione, l’esperienza maturata dimostra come il fuzzing rappresenti una tecnica efficace e pratica per l’analisi della sicurezza del software, anche in contesti reali. L’integrazione di strumenti come AFL e MAGMA consente non solo di individuare vulnerabilità concrete, ma anche di comprendere più a fondo le cause degli errori e di promuovere pratiche di sviluppo più robuste e orientate alla sicurezza.